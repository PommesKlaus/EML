\subsection{Logistische Regression}
\label{LogRegression}
Diskreter Wertebereich der Zielvariablen $y^{(i)}$, idR. endlich und oft auch nur binär ($y^{(i)} \in \{0,1\}$). Klassen haben idR. keine (eindeutige) Ordnung.\\

Einsetzen eines linearen Modells $h_\Theta$ in eine Funktion $g(z)=\frac{1}{1+e^{-z}}$ mit dem Zielbereich $(0,1)$  ergibt sog. \emph{Sigmoid-Funktion}:

\begin{equation*}
    h_\Theta^{\text{logit}}(x) = \frac{1}{1+e^{-(\Theta_0 + \Theta_1x_1 + \dots + \Theta_nx_n)}}
\end{equation*}

Klassifikation durch Schwellwert $0.5$:

\begin{equation*}
    \text{clf}_f(x) = \left\{
        \begin{array}{ll}
            1 & \text{falls } h_\Theta^{\text{logit}}(x) \geq 0.5\\
            0 & \text{falls } h_\Theta^{\text{logit}}(x) < 0.5
        \end{array}
    \right.
\end{equation*}\\

bzw. bei $n$ Klassen diejenige Klasse $k$, für die $h_\Theta^{\text{logit}}(x)_k$ maximal ist. Bewertung mittels der \emph{logistischen Kostenfunktion $L^\text{logit}$}:

\begin{equation*}
    L^\text{logit}(D, f) = -\sum_{i=1}^{m}\left[\underset{a}{\underbrace{y^{(i)}\ln(f(x^{(i)}))}} + \underset{b}{\underbrace{(1-y^{(i)})\ln(1-f(x^{(i)}))}}\right]
\end{equation*}\\

Bei $y=1$ wird $b = 0$, bei $y=0$ wird $a=0$ und es ergibt sich:


\begin{center}
    \begin{tabular}{|c|c|c|}
        \hline
        $f(x)$ & $y$ & $L^\text{logit}(D, f)$\\
        \hline
        1 & 1 & 0\\
        1 & 0 & $\infty$\\
        0 & 0 & 0\\
        0 & 1 & $\infty$\\
        \hline
    \end{tabular}
\end{center}

\textbf{Ziel}: Minimierung der Kostenfunktion, wobei es sich um ein konvexes Optimieungsproblem handelt (d.h. es existiert nur ein globales Minimum). Effiziente Lösung mithilfe numerischer Methoden wie \emph{Gradient Descent} möglich.\\

\textbf{Polynominelle Erweiterung} sowie \textbf{Regularisierung} analog zur \emph{linearen Regression}.\\

\textbf{Evaluation}: Berechnung der \emph{Konfustionsmatrix} und Einsetzen in die \emph{Klassifikationsmetriken}:

\begin{center}
    \begin{tabular}{|c|c|c|}
        \hline
        & $y=1$ & $y=0$\\
        \hline
        $\text{clf}=1$ & TP & FP\\
        $\text{clf}=0$ & FN & TN\\
        \hline
    \end{tabular}
\end{center}

\begin{itemize}
    \item \emph{Accuracy/Genauigkeit}: $\text{acc}(D, \text{clf})=\frac{TP+TN}{TP+TN+FP+FN}\rightarrow$ Verhältnis der korrekt klassifizierten Instanzen zu allen Instanzen; bei ungleicher Klassenrepräsentation nicht geeignet
    \item \emph{Precision}: $\text{prec}(D, \text{clf})=\frac{TP}{TP+FP}\rightarrow$ wie viele der positiv klassifizierten Instanzen sind tatsächlich positiv?
    \item \emph{Recall/Sensitivität}: $\text{rec}(D, \text{clf})=\frac{TP}{TP+FN}\rightarrow$ wie viele Ist-positive Instanzen wurden korrekt klassifiziert?
    \item \emph{F1-Score}: $\text{F1}(D, \text{clf})=2\cdot\frac{\text{Precision}\cdot\text{Recall}}{\text{Precision}+\text{Recall}}\rightarrow$ harmonisches Mittel von Precision und Recall (Gesamtqualität)
\end{itemize}

